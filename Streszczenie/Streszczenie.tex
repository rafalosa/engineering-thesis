\chapter*{Streszczenie}

Na świecie obecnie obserwuje się gwałtowny rozwój przemysłu kosmicznego. Z każdym kolejnym
 rokiem kwota inwestowana w firmy tego sektora zwiększa się, osiągając poziom 9.1 miliarda
  dolarów w roku 2020 \cite{bib:kosmos_raport_kwartalny_2021}. Szacuje się, iż ta kwota
   zostanie przekroczona pod koniec obecnego 2021-go roku
    \cite{bib:kosmos_raport_kwartalny_2021}. Tak gwałtowny wzrost nieuchronnie doprowadzi do
     sytuacji, w których koniecznością będzie również wysyłanie coraz większej liczby ludzi
      w kosmos. To naturalnie zrodzi zapotrzebowanie na technologię produkcji przedmiotów i
       materiałów niezbędnych do życia, w warunkach pozaziemskich, ze względu na wysokie
        koszty wynoszenia ładunku poza Ziemię (mała pojemność kapsuł transportowych).
         Warunki pozaziemskie tj. o niskiej grawitacji bądź mikrograwitacji, dla
          biologicznych struktur wzrostowych np. roślin, można symulować na Ziemi za pomocą
           specjalnych urządzeń - klinostatów. Pozwala to na badanie wpływu warunków
            panujących w kosmosie na owe struktury bez konieczności wynoszenia ich na
             orbitę, co znacznie redukuje koszty badań.\\

Celem obecnej pracy jest rozwój projektu budowy i wykonania klinostatu, którego podstawową
 koncepcję wykonano w ramach projektu Project Based Learning (PBL), w którym uczestniczyłem
  dwa lata temu. W czasie realizacji projektu klinostatu zaprojektowano wiele autorskich
   usprawnień mechanicznych, które również zostały opisane w tej pracy. Główną częścią pracy
    dyplomowej jest projekt i wykonanie systemu kontroli klinostatu, który pozwala dowolnemu
     użytkownikowi na łatwe i intuicyjne sterowanie tym urządzeniem. Cały system składa się
      z trzech modułów - sterownika kontrolera klinostatu, aplikacji pulpitowej oraz
       programu komputera komory uprawnej, służącego do akwizycji danych i przekazywania ich
        do aplikacji za pomocą łącza bezprzewodowego. Program komputera komory oraz
         aplikację pulpitową napisano w języku Python z wykorzystaniem popularnych bibliotek
          numerycznych (scipy, numpy) oraz tworzenia interfejsu (Tkinter). Sterownik klinostatu ze
           względu na implementację na platformie ze znacznie ograniczoną mocą obliczeniową
            oraz ograniczonymi zasobami zaprogramowano w języku C++. \\

W początkowych rozdziałach pracy skupię się na przedstawieniu różnych sposobów symulacji warunków mikrograwitacji. Przedstawione zostaną również podstawowe informacje na temat
 klinostatów. Ma to na celu wprowadzenie czytającego w zasadę działania, konstrukcję oraz
  podział klinostatów, jak i przedstawienie obecnego stanu badań z ich wykorzystaniem. Następnie przybliżony czytelnikowi zostanie wspomniany projekt PBL, budowa zaprojektowanego klinostatu oraz stworzone dla niego usprawnienia. Reszta pracy poświęcona zostanie przedstawieniu stworzonego systemu kontroli od konceptu aż po metody jego implementacji. Praca ta ma również spełniać rolę pewnego rodzaju dokumentacji technicznej wykonanego systemu oraz urządzenia.\\
  

{\bf Słowa kluczowe:} klinostat, mikrograwitacja, oprogramowanie, system, kontrola.

\addcontentsline{toc}{chapter}{Streszczenie}
