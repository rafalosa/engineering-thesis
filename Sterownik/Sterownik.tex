\chapter{Sterownik kontrolera klinostatu}

Jako kontroler klinostatu rozumie się jednostkę odpowiadającą za bezpośrednie sterowanie silnikami oraz pompą wody. Jest to część systemu z którą operator urządzenia nie wchodzi bezpośrednio w interakcję, a sterowana jest ona pośrednio przez interfejs użytkownika aplikacji, poprzez system wbudowanych komend. Kontroler stanowi całkowicie odrębne urządzenie niż komputer z którego sterowany jest klinostat, natomiast nie jest ono w stanie działać samodzielnie bez kontaktu z rdzeniem systemu. Sterownik wymagał implementacji w niskopoziomowym języku programowania, ze względu na wybór platformy z mikroprocesorem typu AVR. Wybrany został w tym celu język C++. W dalszej części pracy do programu kontrolera klinostatu będzie 
\section{Założenia funkcjonalne sterownika}

Poniżej wymieniono funkcjonalności oraz założenia jakie musiał spełniać sterownik kontrolera klinostatu:

\begin{itemize}
	
	\item Sterowanie silników powinno być całkowicie niezależne od reszty funkcjonalności programu. Oznacza to iż odbieranie oraz wysyłanie danych przez port szeregowy powinno odbywać się równolegle z operacjami obsługującymi układ napędowy.
	\item Silniki domyślnie powinny uruchamiać się oraz zatrzymywać ze stałym przyspieszeniem tzn. prędkości obrotowe powinny zmieniać się liniowo podczas rozruchu oraz zatrzymania klinostatu.
	\item Sterownik powinien również niezależnie śledzić czas, który upłynął od rozpoczecia programu. Z pomocą odmierzania czasu obliczana jest objętość wody wpompowanej do komory środowiskowej.

\end{itemize}



\section{Platforma}



\section{Rejestry mikrokontrolera}

\section{Rejestry typu timer}

\section{Przerwania programowe}

\section{Sterowanie silnikami krokowymi}

\section{Komunikacja USB}

\section{Konfiguracja sterownika}