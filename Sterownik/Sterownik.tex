\chapter{Sterownik kontrolera klinostatu}

Jako kontroler klinostatu rozumie się jednostkę odpowiadającą za bezpośrednie sterowanie silnikami oraz pompą wody. Jest to część systemu z którą operator urządzenia nie wchodzi bezpośrednio w interakcję, a sterowana jest ona pośrednio przez interfejs użytkownika aplikacji, poprzez system wbudowanych komend. Kontroler stanowi całkowicie odrębne urządzenie niż komputer z którego sterowany jest klinostat, natomiast nie jest ono w stanie działać samodzielnie bez kontaktu z rdzeniem systemu. Sterownik wymagał implementacji w niskopoziomowym języku programowania, ze względu na wybór platformy z mikroprocesorem typu AVR. Wybrany został w tym celu język C++. W dalszej części pracy do programu kontrolera klinostatu będzie 
\section{Założenia funkcjonalne sterownika}

Poniżej wymieniono funkcjonalności oraz założenia jakie musiał spełniać sterownik kontrolera klinostatu:

\begin{itemize}
	
	\item Sterowanie silników powinno być całkowicie niezależne od reszty funkcjonalności programu. Oznacza to iż odbieranie oraz wysyłanie danych przez port szeregowy powinno odbywać się równolegle z operacjami obsługującymi układ napędowy.
	\item Silniki domyślnie powinny uruchamiać się oraz zatrzymywać ze stałym przyspieszeniem tzn. prędkości obrotowe powinny zmieniać się liniowo podczas rozruchu oraz zatrzymania klinostatu.
	\item Sterownik powinien również niezależnie śledzić czas, który upłynął od rozpoczecia programu. Z pomocą odmierzania czasu obliczana jest objętość wody wpompowanej do komory środowiskowej.

\end{itemize}

COŚ TUTAJ JESZCZE DODAĆ, NA RAZIE PLACEHOLDER.


\section{Platforma}

Wspomniano wcześniej iż w skład elektroniki klinostatu wchodzi płytka rozwojowa Arduino Leonardo. Jednostka mikrokontrolera (ang. \angver{microcontroller unit}, MCU) znajdująca się na tym rodzaju płytki to ATmega32U4 produkowana przez firmę Atmel. Jest to 8-bitowy MCU typu AVR, który w przypadku płytki Arduino taktowany jest za pomocą zewnętrznego oscylatora kwarcowego o częstotliwości \SI{16}{MHz}. Platforma ta została wybrana na etapie projektowym, kierując się jej dostępnością, popularnością, ceną oraz parametrami. Platformy Arduino w odniesieniu do samodzielnych kontrolerów AVR oferują znacznie prostsze metody programowania poprzez udostępnienie wielu bibliotek, które interfejsują z samymi rejestrami mikrokontrolera. Powoduje to efektywnie programowanie wyższego poziomu. Minusem takiego podejścia jest konieczność umieszczania ów bibliotek w pamięci kontrolera, która jest ograniczona. Dodatkowo platformy arduino posiadają stały fragment kodu, który nie jest modyfikowalny przez użytkownika, tzw. bootloader, który również zajmuje część dostępnej pamięci. MCU zdecydowano się programować w czystym języku C++, z uwagi na to iż biblioteki Arduino nie oferowały wymaganej funkcjonalności. Bootloader umieszczony na platformach Arduino umożliwia programowanie przez interfejs USB. W przypadku sterownika klinostatu, bootloader usunięto, a układ programowano poprzez programator USBASP (ang. \angver{USB AVR Serial Programmer}).

\section{Rejestry mikrokontrolera}

W przypadku mikrokontrolerów AVR, kontrolowanie oraz konfiguracja ich peryferiów odbywa się poprzez konfigurowanie wewnętrznych rejestrów. Rejestry mikrokontrolera są po prostu lokalizacjami w pamięci, których wartość można odczytać lub ją zmodyfikować. Część rejestrów MCU wyznacza jego pamięć RAM (ang. \angver{Random Access Memory}), w której przechowywane są tymczasowo wartości zmiennych lub stałych, które potrzebne są w momencie wykonywania programu. Inna część rejestrów to tzw. rejestry specjalne (ang. \angver{Special Function Register}, SFR), których zadaniem jest wcześniej wspomniana kontrola oraz konfiguracja elementów mikrokontrolera.

\section{Rejestry układu czasowo-licznikowego}

Jednym z ważnych dla działania klinostatu typów SFR są rejestry odpowiedzialne za konfigurację oraz kontrolę układów czasowo-licznikowych. Układy te cyklicznie inkrementują wartość przechowywaną w pewnych rejestrze wraz z cyklami zegara MCU. Znając jego częstotliwość taktowania pozwala to efektywnie odmierzać czas programu. Atmega32U4 posiada 4 rejestry tego typu: 
\begin{itemize}
	\item jeden rejestr 8-bitowy, Timer/Counter 0
	\item dwa rejestry 16-bitowe, Timer/Counter 1 i 3
	\item jeden rejestr 10-bitowy o dużej szybkości, Timer/Counter 4
\end{itemize}

W sterowniku wykorzystane zostały 3 z dostępnych rejestrów układów czasowo-licznikowych. Dwa rejestry 16-bitowe przeznaczone zostały do wyznaczania interwałów czasowych między kolejnymi krokami silników, natomiast rejestr 8-bitowy przeznaczony został na odmierzanie czasu programu. Układy czasowo-licznikowe nie muszą inkrementować wartości w rejestrze co każdy cykl procesora, mogą to robić co określoną ilość cykli. Taką ilość cykli wyznacza prescaler, który również jest konfigurowany przez odpowiednie rejestry. Wybranym trybem pracy układów czasowo-licznikowych jest tryb CTC (ang. \angver{Clear Timer on Compare Match}), który zeruje rejestr wartości układu licznika kiedy osiąga on ustaloną programowo wartość. Pozwala to na sterowanie długością interwałów pomiędzy impulsami za pomocą operacji na wartościach w  rejestrach OCRnA lub OCRnB (ang. \angver{Output Compare Register}), gdzie $n$ jest numerem układu. Konfiguracja owych układów zostanie opisana w dalszych podrozdziałach.

\section{Przerwania programowe}

Przerwania programowe są kolejną, kluczową funkcją mikrokontrolera dla spełnienia założeń sterownika klinostatu. Przerwania programowe umożliwiają przerwanie wykonywania programu w dowolnym jego miejscu i tymczasowe przekazanie kontroli do specjalnej części kodu zwanej procedurą obsługi przerwania (ang. \angver{Interrupt Service Routine}, ISR). Umożliwia to wykonywanie kluczowych operacji niezależnie od tego co w danej chwili wykonuje główny program kontrolera. Ważne jest, aby zakończyć operacje związane z ISR we względnie krótkim czasie, aby uniknąć wyzwolenia kolejnego przerwania w trakcie obsługi obecnego. Może się również zdarzyć iż obsługa przerwań zostaje wyłączona na czas wykonania ISR, co może sprawić iż zostanie ominięte zdarzenie kluczowe dla kontroli systemu. Przerwania mogą zostać wywołane przez wiele sygnałów począwszy od flag dokonania konwersji przez wewnętrzny przetwornik analogowo-cyfrowy, po wysoki stan napięciowy przyłożony do jednego z terminali mikrokontrolera. W przypadku sterownika klinostatu przerwania wywoływane są poprzez osiągnięcie wartości zadanych w OCRnA przez rejestry TCNTn (ilość zliczeń licznika), a procedury obsługi przerwań odpowiedzialne są za wykonanie pojedynczego kroku silnika krokowego oraz za inkrementowanie liczby milisekund upłyniętych od momentu uruchomienia programu.

\section{Sterowanie silnikami krokowymi}

Jak wspomniano w założeniach programu, wymogiem była możliwość liniowej zmiany prędkości obrotowej silników krokowych. Pomaga to odciążyć silniki podczas fazy rozruchowej klinostatu, ograniczając gubione kroki. Szybkość obrotów silników krokowych zależy od częstotliwości wykonywanych kroków. Częstotliwość kroków zależna jest od czasu mijającego pomiędzy kolejnymi krokami. To właśnie ten interwał jest wielkością, która jest bezpośrednio sterowana z poziomu programu kontrolera.

\section{Komunikacja USB}

\section{Konfiguracja sterownika}