 %%%%%%%%%%%%%%%%%%%%%%%%%%%%%%%%%%%%%%%%%%
%                                        %
% Szablon pracy dyplomowej inżynierskiej % 
%                                        %
%%%%%%%%%%%%%%%%%%%%%%%%%%%%%%%%%%%%%%%%%%



\documentclass[a4paper,twoside,12pt]{book}
\usepackage[utf8]{inputenc}                                      
\usepackage[T1]{fontenc}  
\usepackage{amsmath,amsfonts,amssymb,amsthm}
\usepackage[british,polish]{babel} 
\usepackage{indentfirst}
\usepackage{lmodern}
\usepackage{graphicx} 
\usepackage{hyperref}
\usepackage{booktabs}
%\usepackage{tikz}
%\usepackage{pgfplots}
\usepackage{mathtools}
\usepackage{geometry}
\usepackage[page]{appendix} % toc,
\renewcommand{\appendixtocname}{Dodatki}
\renewcommand{\appendixpagename}{Dodatki}
\renewcommand{\appendixname}{Dodatek}

\usepackage{setspace}
\onehalfspacing


\frenchspacing

\usepackage{listings}
\lstset{
	language={},
	basicstyle=\ttfamily,
	keywordstyle=\lst@ifdisplaystyle\color{blue}\fi,
	commentstyle=\color{gray}
}

%%%%%%%%%

%%%% TODO LIST GENERATOR %%%%%%%%%

%\usepackage{tikz}
%\usepackage{manfnt}   % dangerous sign 
\usepackage{color}
\definecolor{brickred}      {cmyk}{0   , 0.89, 0.94, 0.28}

\makeatletter \newcommand \kslistofremarks{\section*{Uwagi} \@starttoc{rks}}
  \newcommand\l@uwagas[2]
    {\par\noindent \textbf{#2:} %\parbox{10cm}
{#1}\par} \makeatother


\newcommand{\ksremark}[1]{%
{%\marginpar{\textdbend}
{\color{brickred}{[#1]}}}%
\addcontentsline{rks}{uwagas}{\protect{#1}}%
}

\newcommand{\comma}{\ksremark{przecinek}}
\newcommand{\nocomma}{\ksremark{bez przecinka}}
\newcommand{\styl}{\ksremark{styl}}
\newcommand{\ortografia}{\ksremark{ortografia}}
\newcommand{\fleksja}{\ksremark{fleksja}}
\newcommand{\pauza}{\ksremark{pauza `--', nie dywiz `-'}}
\newcommand{\kolokwializm}{\ksremark{kolokwializm}}

%%%%%%%%%%%%%% END OF TODO LIST GENERATOR %%%%%%%%%%%

%%%%%%%%%%%% ZYWA PAGINA %%%%%%%%%%%%%%%
% brak kapitalizacji zywej paginy
\usepackage{fancyhdr}
\pagestyle{fancy}
\fancyhf{}
\fancyhead[LO]{\nouppercase{\it\rightmark}}
\fancyhead[RE]{\nouppercase{\it\leftmark}}
\fancyhead[LE,RO]{\it\thepage}


\fancypagestyle{tylkoNumeryStron}{%
   \fancyhf{} 
   \fancyhead[LE,RO]{\it\thepage}
}

\fancypagestyle{NumeryStronNazwyRozdzialow}{%
   \fancyhf{} 
   \fancyhead[LO]{\nouppercase{\it\rightmark}}
   \fancyhead[RE]{\nouppercase{\it\leftmark}}
   \fancyhead[LE,RO]{\it\thepage}
}


%%%%%%%%%%%%% OBCE WTRETY  
\newcommand{\obcy}[1]{\emph{#1}}
\newcommand{\ang}[1]{{\selectlanguage{british}\obcy{#1}}}
%%%%%%%%%%%%%%%%%%%%%%%%%%%%%

% polskie oznaczenia funkcji matematycznych
\renewcommand{\tan}{\operatorname {tg}}
\renewcommand{\log}{\operatorname {lg}}

% jeszcze jakies drobiazgi

\newcounter{stronyPozaNumeracja}

\newcommand{\hcancel}[1]{%
    \tikz[baseline=(tocancel.base)]{
        \node[inner sep=0pt,outer sep=0pt] (tocancel) {#1};
        \draw[red] (tocancel.south west) -- (tocancel.north east);
    }%
}%

\newcommand{\miesiac}{%
  \ifcase\the\month
  \or styczeń% 1
  \or luty% 2
  \or marzec% 3
  \or kwiecień% 4
  \or maj% 5
  \or czerwiec% 6
  \or lipiec% 7
  \or sierpień% 8
  \or wrzesień% 9
  \or październik% 10
  \or listopad% 11
  \or grudzień% 12
  \fi}


%%%%%%%%%%%%%%%%%%%%%%%%%%%%%%%%%%%%%%%%%%%%%%
% Helvetica font macros for the title page:
\newcommand{\headerfont}{\fontfamily{phv}\fontsize{18}{18}\bfseries\scshape\selectfont}
\newcommand{\titlefont}{\fontfamily{phv}\fontsize{18}{18}\selectfont}
\newcommand{\otherfont}{\fontfamily{phv}\fontsize{14}{14}\selectfont}

%%%%%%%%%%%%%%%%%%%%%%%%%%%%%%%%%%%%%%%%%%%%%%
%%%%%%%%%%%%%%%%%%%%%%%%%%%%%%%%%%%%%%%%%%%%%%
%%%%%%%%%%%%%%%%%%%%%%%%%%%%%%%%%%%%%%%%%%%%%%
%%%%%%%%%%%%%%%%%%%%%%%%%%%%%%%%%%%%%%%%%%%%%%
%%%%%%%%%%%%%%%%%%%%%%%%%%%%%%%%%%%%%%%%%%%%%%
%%%%%%%%%%%%%%%%%%%%%%%%%%%%%%%%%%%%%%%%%%%%%%
%%%%%%%%%%%%%%%%%%%%%%%%%%%%%%%%%%%%%%%%%%%%%%


\newcommand{\autor}{Rafał Osadnik}
\newcommand{\promotor}{dr hab. Tomasz Błachowicz}
\newcommand{\tytul}{Rekonstrukcja i analiza stanu nieważkości w warunkach laboratoryjnych z użyciem klinostatu}
\newcommand{\polsl}{Politechnika Śląska}
\newcommand{\wydzial}{Instytut Fizyki}
\newcommand{\kierunek}{Kierunek: Fizyka Techniczna}

\begin{document}
%\kslistofremarks 
	
%%%%%%%%%%%%%%%%%%  STRONA TYTULOWA %%%%%%%%%%%%%%%%%%%
\pagestyle{empty}
{
	\newgeometry{top=2.5cm,%
	             bottom=2.5cm,%
	             left=3cm,
	             right=2.5cm}
	\sffamily
	\rule{0cm}{0cm}
	
	\begin{center}
	\includegraphics[width=45mm]{logo.jpg}
	\end{center} 
	\vspace{1cm}
	\begin{center}
	\headerfont \polsl
	\end{center}
	\begin{center}
	\headerfont \wydzial
	\end{center}
	\begin{center}
	\headerfont \kierunek
	\end{center}
	\vfill
	\begin{center}
	\titlefont Praca dyplomowa inżynierska
	\end{center}
	\vfill
	
	\begin{center}
	\otherfont \tytul\par
	\end{center}
	
	\vfill
	
	\vfill
	 
	\noindent\vbox
	{
		\hbox{\otherfont Autor: \autor}
		\vspace{12pt}
		\hbox{\otherfont Promotor: \promotor}
		\vspace{12pt}
	}
	\vfill 
 
   \begin{center}
   \otherfont Gliwice,  \miesiac\ \the\year
   \end{center}	
	\restoregeometry
}
  

\cleardoublepage
 

\rmfamily
\normalfont



%%%%%%%%%%%%%%%%%% SPIS TRESCI %%%%%%%%%%%%%%%%%%%%%%
\pagenumbering{Roman}
\pagestyle{tylkoNumeryStron}
\tableofcontents

%%%%%%%%%%%%%%%%%%%%%%%%%%%%%%%%%%%%%%%%%%%%%%%%%%%%%
\setcounter{stronyPozaNumeracja}{\value{page}}
\mainmatter

\pagestyle{empty}


\chapter*{Streszczenie}

Na świecie obecnie obserwuje się gwałtowny rozwój przemysłu kosmicznego. \\Z każdym kolejnym
 rokiem kwota inwestowana w firmy tego sektora zwiększa się, osiągając poziom 9.1 miliarda
  dolarów w roku 2020 \cite{bib:kosmos_raport_kwartalny_2021}. Szacuje się, iż ta kwota
   zostanie przekroczona pod koniec obecnego 2021-go roku
    \cite{bib:kosmos_raport_kwartalny_2021}. Tak gwałtowny wzrost nieuchronnie doprowadzi do
     sytuacji, w których koniecznością będzie również wysyłanie coraz większej liczby ludzi
      w kosmos. To naturalnie zrodzi zapotrzebowanie na technologię produkcji przedmiotów i
       materiałów niezbędnych do życia, w warunkach pozaziemskich, ze względu na wysokie
        koszty wynoszenia ładunku poza Ziemię (mała pojemność kapsuł transportowych).
         Warunki pozaziemskie tj. o niskiej grawitacji bądź mikrograwitacji, dla
          biologicznych struktur wzrostowych np. roślin, można symulować na Ziemi za pomocą
           specjalnych urządzeń - klinostatów. Pozwala to na badanie wpływu warunków
            panujących w kosmosie na owe struktury bez konieczności wynoszenia ich na
             orbitę, co znacznie redukuje koszty badań.\\

Celem obecnej pracy jest rozwój projektu budowy i wykonania klinostatu, którego podstawową
 koncepcję wykonano w ramach projektu Project Based Learning (PBL), w którym uczestniczyłem
  dwa lata temu. W czasie realizacji projektu klinostatu zaprojektowano wiele autorskich
   usprawnień mechanicznych, które również zostały opisane w tej pracy. Główną częścią pracy
    dyplomowej jest projekt i wykonanie systemu kontroli klinostatu, który pozwala dowolnemu
     użytkownikowi na łatwe i intuicyjne sterowanie tym urządzeniem. Cały system składa się
      z trzech modułów - sterownika kontrolera klinostatu, aplikacji pulpitowej oraz
       programu komputera komory uprawnej, służącego do akwizycji danych i przekazywania ich
        do aplikacji za pomocą łącza bezprzewodowego. Program komputera komory oraz
         aplikację pulpitową napisano w języku Python z wykorzystaniem popularnych bibliotek
          numerycznych (scipy, numpy) oraz tworzenia interfejsu (Tkinter). Sterownik klinostatu ze
           względu na implementację na platformie ze znacznie ograniczoną mocą obliczeniową
            oraz ograniczonymi zasobami zaprogramowano w języku C++. \\

W początkowych rozdziałach pracy skupię się na przedstawieniu różnych sposobów symulacji warunków mikrograwitacji. Przedstawione zostaną również podstawowe informacje na temat
 klinostatów. Ma to na celu wprowadzenie czytającego \\w zasadę działania, konstrukcję oraz
  podział klinostatów, jak i przedstawienie obecnego stanu badań z ich wykorzystaniem. Następnie przybliżony czytelnikowi zostanie wspomniany projekt PBL, budowa zaprojektowanego klinostatu oraz stworzone dla niego usprawnienia. Reszta pracy poświęcona zostanie przedstawieniu stworzonego systemu kontroli od konceptu aż po metody jego implementacji. Praca \\ta ma również spełniać rolę pewnego rodzaju dokumentacji technicznej wykonanego systemu oraz urządzenia.\\
  

{\bf Słowa kluczowe:} klinostat, mikrograwitacja, oprogramowanie, system, kontrola.

\addcontentsline{toc}{chapter}{Streszczenie}



\cleardoublepage

\pagestyle{NumeryStronNazwyRozdzialow}
%%%%%%%%%%%%%% wlasciwa tresc pracy %%%%%%%%%%%%%%%%%


\chapter{Wstęp}

Placeholder na razie, bo nie mam w sumie pomysłu xd.

\section{Grawitropizm}

Placeholder aż znajdę sensowną literaturę

\section{Klinostat}



\section{Zasada działania klinostatu}

\section{Rodzaje klinostatów}

\section{Obecne rozwiązania}

\graphicspath{{./PBL/images}}

\chapter{Projekt PBL}

Uczenie poprzez realizację projektów (ang. \angver{Project Based Learning}, PBL) jest metodą uczenia 

\section{Cel projektu} \label{cel_projektu}

Celem wspomnianego projektu było zaprojektowanie układu laboratoryjnego, składającego się z
 klatki Helmholtza oraz umieszczonego wewnątrz klinostatu trójwymiarowego. Docelowo taki układ
  pozwalałby na przeprowadzanie badań nad rozwojem roślin w warunkach symulowanej
   mikrograwitacji bez ziemskiego pola magnetycznego. Klatka Helmholtza jest złożeniem trzech
    par cewek Helmholtza w takiej konfiguracji aby osie magnetyczne każdej z par było
     prostopadłe do pozostałych. Pozwala to wytworzyć wektor indukjcji magnetycznej o dowolnym
      kierunku przestrzennym i bardzo wysokiej jednorodności. Wysoka jednorodność jest kluczowa
       aby generowany wektor indukcji magnetycznej był stały w objętości przeprowadzanego
        eksperymentu. W związku z wymaganiami projektu, wewnętrzna konstrukcja klinostatu
         musiała zostać wykonana tak, aby mieć jak najmniejszy wpływ na panujące wewnątrz
          warunki magnetyczne. Naturalnie rodzi to zagadnienie wyboru materiałów konstrukcyjnych
           klinostatu, które nie powinny mieć właściwości ferromagnetycznych, powinny mieć
            względnie niską przewodność elektryczną w celu eliminacji indukowanych prądów
             wirowych oraz dodatkowym ich atutem będzie niska podatność magnetyczna. Oprócz
              właściwości magnetycznych materiałów istnotne są też możliwości ich obróbki
               termicznej oraz mechanicznej, a również ich forma, co później ma wpływ na
                prostotę montażu urządzenia. Oprócz samych wymagań materiałowych oraz
                 konstrukcyjnych, należało również zapewnić optymalne warunki wzrostowe obiektów
                  badawczych, należało określić odpowiedni rozmiar cewek, aby objętość o wysokim
                   stopniu jednorodności magnetycznej była odpowiednio duża oraz wiele innych.
                    Taki szereg wymagań stworzył bardzo ciekawe i multidyscyplinarne wyzwanie
                     inżynieryjne, które wymagało użycia oraz w niektórych przypadkach
                      stworzenia wielu środowisk symulacyjnych w celu jego rozwiązania. Projekt
                       wykonywany był w zespole 5-cio osobowym na przestrzeni jednego semestru.
                        Powstały na rzecz projektu model komputerowy układu klinostatu wraz z
                         klatką Helmholtza przedstawiony został na Rys.
                          \ref{fig:klatka_helmholtza}.
                          

\begin{figure}
	\centering
	\includegraphics[scale=0.5]{klinostat_klatka}
	\caption{Projekt klatki Helmholtza z klinostatem.} 
	\caption*{Źródło: opracowanie własne}
	\label{fig:klatka_helmholtza}
\end{figure}

\section{Konstrukcja klinostatu}

Ten podrozdział poświęcony został krótkiemu opisowi konstrukcji samego klinostatu
 zaprojektowanego w ramach PBL. Zaprojektowane urządzenie jest klinostatem o dwóch stopniach
  swobody, każdy stopień posiada swój osobny napęd. Oznacza to iż jest to maszyna RPM, natomiast
   jest możliwość uruchomienia go w trybach klinostatu 2-D oraz 3-D jak opisano w podrozdziale
    \ref{klinostat3d}. Zewnętrzna rama klinostatu wykonana została z rur polipropylenowych (PP)
     stabilizowanych włóknem szklanym. Odcinki rur wraz z kształtkami 90$^\circ$ oraz
      czwórnikami zostały połączone metodą polifuzji termicznej (zgrzewanie). Tego typu rury
       wykorzystywane są w instalacjach centralnego ogrzewania, dzięki czemu ich koszt jest
        niski. Konstrukcję ramy przedstawiono na Rys. \ref{fig:rama_klinostatu}. Przeprowadzone
         analizy elementów skończonych (FEM), wskazały iż materiał ten posiada wystarczającą
          wytrzymałość aby wykorzystać go jako element strukturalny ramy klinostatu. Jako wał
           obrotowy wykorzystano rurę aluminiową o średnicy \SI{12}{mm}. Wykorzystanie
            konwencjonalnych łożysk kulowych w konstrukcji klinostatu nie było wskazane ze
             względu na wymagania opisane w podrozdziale \ref{cel_projektu}. Z tego powodu każdy
              z interfejsów obrotowych stanowią tuleje ślizgowe wykonane z materiału
               iglidur$\copyright$, zaprojektowanego przez firmę IGUS. Większość pozostałych
                elementów została wykonana w technologii druku przestrzennego z materiału PETG.
                 Konstrukcję jednego z czterech czwórników ramy przedstawiono na Rys.
                  \ref{fig:czwórnik}. Wewnętrzny stopień swobody składa się z kulistej komory
                   środowiskowej, która posiada swój dedykowany komputer o niskiej mocy,
                    monitorujący panujące wewnątrz warunki, oraz sterujący oświetleniem.
                     Zasilanie do wnętrza komory doprowadzone jest przez szereg złącz
                      ślizgowych, a przewody poprowadzone są wewnątrz ramy klinostatu. Podobne
                       rozwiązanie wykorzystano w obwodzie doprowadzjącym wodę, przewody
                        prowadzone są wewnątrz ramy klinostatu, a przy elementach obrotowych
                         zastosowano kolanka obrotowe. Rozmiar komory oraz typ i moc oświetlenia
                          zostały wyznaczone na podstawie wyników symulacji stworzonych na rzecz
                           projektu. Jej projekt przedstawiony został na Rys. \ref{fig:komora}.
                            Mocowania śrubowe zostały w większości zrealizowane za pomocą śrub
                             poliamidowych (PA). Komora została zaprojektowana tak, aby
                              umożliwić operatorowi dostęp do jej wnętrza bez konieczności jej
                               demontażu z ramy klinostatu.
                               

\begin{figure}
	\centering
	
	\begin{subfigure}[b]{.49\textwidth}
		\centering
		\includegraphics[width=\textwidth]{rama_40_aisass}
		\caption{Rama klinostatu} 
		\label{fig:rama_klinostatu}
	\end{subfigure}
	\hfill%
	\begin{subfigure}[b]{.49\textwidth}
		\centering
		\includegraphics[width=\textwidth]{2_diss}
		\caption{Czwórnik ramy klinostatu.} 
		
		\label{fig:czwórnik}
	\end{subfigure}\vspace{15mm}%
	
	\begin{subfigure}{.8\textwidth}
		\centering
		\includegraphics[scale=0.3]{Komora_tweaked_colors_exploded}
		\caption{Komora środowiskowa.} 
		\label{fig:komora}
	\end{subfigure}

	\caption{Przykładowe części modelu komputerowego projektu.}
	\caption*{Źródło: opracowanie własne}
	
\end{figure}


\section{Modyfikacje projektu}

Jako, że projekt rozpoczął się na początku pandemii COVID-19, niemożliwe było prowadzenie prac
 konstrukcyjnych równolegle z postępem prac modelowych. Spowodowało to zakończenie się projektu
  na fazie ukończonego modelu komputerowego. Konstrukcję całego urządzenia rozpoczęto rok po
   zakończeniu się projektu PBL w zespole dwuosobowym w którego skład wchodziłem. Podczas
    konstrukcji napotkano wiele problemów, które nie zostały przewidziane na etapie prac
     projektowych i wymagały stworzenia nowych elementów urządzenia, bądź zmodyfikowania już
      istniejących komponentów. Oprócz tego wymagane było również stworzenie metod
       konstrukcyjnych tak aby zachować jego kluczowe cechy, oraz aby mogło ono zostać wykonane
        bez użycia kosztownych metod obróbki takich jak wspomagana numerycznie obróbka maszynowa
         (ang. \angver{Computerized Numerical Control}, CNC). Takich poprawek stworzono bardzo
          wiele na drodze konstrukcji urządzenia, natomiast w tym podrozdziale zostaną opisane
           dwie najbardziej kluczowe dla jego poprawnego działania.
           

\subsection{Przeciwwaga ramy klinostatu}

Podczas pierwszych prób uruchomieniowych klinostatu napotkano okresowo pojawiający się moment
 siły, który hamował ruch obrotowy urządzenia przez zbyt duże obciążenie układu napędowego. Za
  jedno ze źródeł tego problemu zidentyfikowano asymetrię obciążenia ramy klinostatu, powodowaną
   układem zmiany kierunku pasa układu napędowego komory środowiskowej, który ulokowany jest na
    rogu ramy. W tym celu na przeciwległym rogu zamontowano przeciwwagę, które generowała moment
     siły o tym samym kierunku, natomiast o przeciwnym zwrocie. Przeciwwaga składa się z
      drukowanego uchwytu oraz kawałka rury PP, która wypełniona jest piaskiem w celu
       zapewnienia odpowiedniej masy. Skonstruowana oraz zamontowana przeciwwaga przedstawiona
        została na Rys. \ref{fig:przeciwwaga}.

\begin{figure}[ht]
	\centering
	\setlength{\fboxsep}{0pt}
	\setlength{\fboxrule}{1pt}
	\fbox{\includegraphics[scale=0.06]{przeciwwaga}}
	\caption{Zamontowana przeciwwaga.} 
	\caption*{Źródło: opracowanie własne}
	\label{fig:przeciwwaga}
\end{figure}

\subsection{Przekładnie walcowe}

Drugą bardzo istotną dla działania klinostatu modfikacją, był projekt i konstrukcja dwóch
 przekładni walcowych, które zwiększają dostępny moment obrotowy układu napędowego. Z uwagi na
  to iż klinostat przeznaczony będzie przedewszystkim do badań nad organizmami roślinnymi, jego
   prędkość obrotowa powinna mieścić się w zakresie od 1 do 2 RPM. W pierwotnej konfiguracji
    klinostat był w stanie osiągnąć znacznie większe prędkości obrotowe, co umożliwiło
     zastosowanie wspomnianych przekładni. Zaprojektowanie przekładnie stosują redukcję obrotów
      silników 4:1. Pozwala to osiągnąć czterokrotnie wyższy moment obrotowy przed głównym kołem
       napędowym klinostatu. Model przekładni widoczny jest na Rys. \ref{fig:projekt
       	 przekładni}. Przekładnia składa się z dwóch kół zębatych o średnicach kolejno $WSTAWIĆ$
         i $WSTAWIĆ$ oraz ilości zębów odpowiednio $WSTAWIĆ$ i $WSTAWIĆ$. Duże koło zębate jest
          łożyskowane w dwóch miejscach przez konwencjonalne łożyska kulowe, ze względu na to iż
           przekładnie znajdują się w wystarczająco dużej odległości od objętości jednorodnego
            pola magnetycznego. Na małym kole zębatym umieszczono dodatkowo wpusty umożliwiajace
             montaż enkoderów, jeśli zajdzie taka potrzeba. Całość zamknięta została w korpusie
              odpowiedzialnym za ochronę kół przez pyłem oraz zapobiegającym wydostaniu się
               smaru na zewnątrz. Skonstruowana przekładna z nałożonym kołem pasowym
                przedstawiona jest na Rys. \ref{fig:gotowa przekładnia}. Zastosowanie przekładni
                 znacznie poprawiło działanie klinostatu, przy jednoczesnym zachowaniu wymaganej
                  prędkości obrotowej urządzenia.       


\begin{figure}
	\centering
	
	\begin{subfigure}[b]{.49\textwidth}
		\centering
		\includegraphics[width=\textwidth]{rama_40_aisass}
		\caption{Projekt przekładni} 
		\label{fig:projekt przekładni}
	\end{subfigure}
	\hfill%
	\begin{subfigure}[b]{.49\textwidth}
		\centering
		\includegraphics[width=\textwidth]{2_diss}
		\caption{Skonstruowana przekładnia.} 
		
		\label{fig:gotowa przekładnia}
	\end{subfigure}

	\caption{Przekładnie klinostatu.}
	\caption*{Źródło: opracowanie własne}
	
\end{figure}



\section{Napęd klinostatu}

\section{Elektronika układu sterowania}

\chapter{Sterownik kontrolera klinostatu}

Jako kontroler klinostatu rozumie się jednostkę odpowiadającą za bezpośrednie sterowanie silnikami oraz pompą wody. Jest to część systemu z którą operator urządzenia nie wchodzi bezpośrednio w interakcję, a sterowana jest ona pośrednio przez interfejs użytkownika aplikacji, poprzez system wbudowanych komend. Kontroler stanowi całkowicie odrębne urządzenie niż komputer z którego sterowany jest klinostat, natomiast nie jest ono w stanie działać samodzielnie bez kontaktu z rdzeniem systemu. Sterownik wymagał implementacji w niskopoziomowym języku programowania, ze względu na wybór platformy z mikroprocesorem typu AVR. Wybrany został w tym celu język C++. W dalszej części pracy do programu kontrolera klinostatu będzie 
\section{Założenia funkcjonalne sterownika}

Poniżej wymieniono funkcjonalności oraz założenia jakie musiał spełniać sterownik kontrolera klinostatu:

\begin{itemize}
	
	\item Sterowanie silników powinno być całkowicie niezależne od reszty funkcjonalności programu. Oznacza to iż odbieranie oraz wysyłanie danych przez port szeregowy powinno odbywać się równolegle z operacjami obsługującymi układ napędowy.
	\item Silniki domyślnie powinny uruchamiać się oraz zatrzymywać ze stałym przyspieszeniem tzn. prędkości obrotowe powinny zmieniać się liniowo podczas rozruchu oraz zatrzymania klinostatu.
	\item Sterownik powinien również niezależnie śledzić czas, który upłynął od rozpoczecia programu. Z pomocą odmierzania czasu obliczana jest objętość wody wpompowanej do komory środowiskowej.

\end{itemize}



\section{Platforma}



\section{Rejestry mikrokontrolera}

\section{Rejestry typu timer}

\section{Przerwania programowe}

\section{Sterowanie silnikami krokowymi}

\section{Komunikacja USB}

\section{Konfiguracja sterownika}

\graphicspath{{./Aplikacja/images}}

\chapter{Aplikacja pulpitowa}

Aplikacja będąca obiektem tego podrozdziału stanowi rdzeń systemu kontroli, dzięki któremu oparator urządzenia jest nim w stanie sterować oraz zbierać z jego pomocą dane. Zawiera ona moduły do komunikacji ze wszystkimi częściami urządzenia, oraz zestaw wykresów aktualizowanych w czasie rzeczywistym. Pomimo względnie prostych funkcji, aplikacja posiada złożoną budowę ze względu na rówoległą obsługę różnych dróg komunikacji. Budowa ta w szczegółach opisana zostanie  w dalszej części tego rozdziału.

\section{Rola i założenia aplikacji}

Aplikacja była rozwijana mając na uwadze poniższe założenia:
\begin{itemize}
	\item Aplikacja powinna umożliwiać oparatorowi zadanie prędkości obu stopni swobody klinostatu oraz objętości wody, która ma zostać dostarczona co wskazany przez operatora interwał czasowy.
	\item Aplikacja powinna dawać dostęp do danych o orientacji komory względem wektora grawitacji, uśrednionej grawitacji i temperatury z ostatnich 60 sekund. Dane odnośnie wilgotności podłoża z uwagi na korozję czujnika przeprowadzane są znacznie rzadziej. Dodatkowo na jednym z wykresów przedstawiana ma być transformata Fouriera sygnału uzyskanego z akcelerometru.
	\item Aplikacja powinna posiadać podstawowe zabezpieczenia, aby uniemożliwić operatorowi zawieszenie urządzenia poprzez nieświadome wykonanie operacji niedozwolonych. Powinny również znaleźć się w niej zabezpieczenia obsługujące błędy generowane przez fizyczne odłączenie klinostatu w momencie gdy program zakłada iż jest on podłączony.
	\item Aplikacja powinna w tle posiadać uruchomiony moduł komunikacji bezprzewodowej z komorą środowiskową. Połączenie to służyć ma wymianie danych oraz nastawie intensywności oświetlenia.
	\item Oprator powinien mieć możliwość zapisania zebranych wyników pomiarów do pliku .csv.
	\item Powinien istnieć podstawowy system komunikatów, który będzie informować operatora o rezultatach jego operacji.
\end{itemize}

\section{Wybrany język i biblioteki}

W celu implementacji aplikacji wybrany został język Python. Wybór ten motywowany był moją dobrą znajomością tego języka, oraz poprzednim doświadczeniem w tworzeniu aplikacji z graficznym interfejsem użytkownika w tym języku. Interfejs został stworzony z użyciem biblioteki \textbf{Tkinter}, która oferuje względnie proste tworzenie takich aplikacji w konwencji programowania obiektowego, z której to intensywnie korzystano podczas rozwoju programu. Do realizacji łącza bezprzewodowego z komorą środowiskową wykorzystano bibliotekę \textbf{socket}, pozwalającą na utworzenie takiego łącza w lokalnej sieci poprzez protokół TCP (ang. \angver{Transmission Control Protocl}). Do działania programu niezbędna okazała się biblioteka \textbf{threading}, pozwalająca na współbieżne wykonywanie różnych jego części. Transformacja Fouriera liczona jest 15 razy na sekundę z użyciem biblioteki \textbf{scipy}, a obliczenia te są zrównoleglone za pomocą biblioteki \textbf{multiprocessing} w celu ich przyspieszenia. Wykresy tworzone są wykorzystując popularną bibliotekę \textbf{matplotlib}. Oprócz tego wykorzystano również wiele innych modułów takich jak \textbf{pyserial}, \textbf{yaml}, \textbf{queue} czy \textbf{numpy} w różnych mniej znaczących celach.

\section{Podział programu na wątki}

Cała aplikacja składa się w jednym momencie z trzech rodzajów wątków:
\begin{itemize}
	\item Wątku głównego - odpowiedzialny za działanie całej aplikacji oraz tworzenie wykresów.
	\item Wątku serwera TCP - odpowiedzialny za obsługę przychodzących pakietów danych przez websocket oraz wysyłanie odpowiedzi.
	\item Wątków obsługi portu szeregowego - tworzony jedynie tymczasowo w momencie gdy należy wysłać i/lub odebrać komendę przez magistralę USB z klinostatem.
\end{itemize}
Docelowo wątek główny miał być odpowiedzialny wyłącznie za uruchamianie innych wątków oraz działanie aplikacji, natomiast tworzenie i akutualizacja wykresów miała być rezultatem działania innego, dodatkowego wątku. Takie rozwiązanie okazało się niemożliwe, ze względu na konstrukcję biblioteki matplotlib, która uniemożliwia tworzenie wykresów przez wątki inne niż główny. Związane jest to z tym iż biblioteka ta nie została stworzona w konwencji \angver{threadsafe} - bezpiecznej w działaniach między wątkami. Przy programowaniu wielowątkowym należy zwracać uwagę na występowanie tzw. wyścigów (ang. \angver{race condition}), które występują w momencie gdy przynajmniej dwa, różne wątki próbują uzyskać dostęp do tej samej, dzielonej zmiennej. Występowanie takich wyścigów jest znane w bibliotece matplotlib, a więc wykorzystywana może być ona wyłącznie w obrębie wątku głównego. Rozwiązaniem problemu wyścigów jest wykorzystanie różnego rodzaju struktur synchronizacyjnych takich jak semafory czy zamki, które zapobiegają ich występowaniu. Z tego rodzaju struktur korzysta również aplikacja.

Wątek serwera TCP jest uruchamiany przez użytkownika za pomocą odpowiedniego przycisku. Pozostaje on uruchomiony w tle aż do momentu, w którym operator postanowi zakończyć komunikację. Jego rolą jest odbieranie danych pomiarowych wysłanych z komputera komory środowiskowej oraz odesłanie odpowiedzi zawierającej informacje o obecnie ustawionym poziomie oświetlenia. Ze względu na to iż dane te muszą zostać przekazane do wątku głównego, wymiana ta zachodzi poprzez wykorzystanie struktur \textbf{Queue} z biblioteki \textbf{queue}, które zapobiegają wyścigom pomiędzy wątkami.

Ostatnim z wątków występujących w programie jest wątek odpowiedzialny za obsługę portu szeregowego, realizującego komunikację z kontrolerem klinostatu. Jego zadaniem jest uzyskanie dostępu do wcześniej utworzonego połączenia szeregowego, wysłanie pożądanej komendy, a następnie uzyskanie odpowiedzi ze strony klinostatu. Takich wątków w jednej chwili może istnieć kilka, ponieważ w programie zachodzą również zautomatyzowane procesy, które wysyłają komendy do klinostatu. Natomiast dostęp do połączenia w tym samym czasie może uzyskać wyłącznie jeden z nich ze względu na wykorzystanie struktury synchronizacyjnej \textbf{lock}. Powoduje to efektywne kolejkowanie operacji wykonywanych na porcie szeregowym. W przeciwnym wypadku program zwróciłby błąd, gdy połączenie z klinostatem zostałoby otwarte po raz drugi. Schemat struktury wątkowej przedstawiony został na Rys. \ref{fig:watki}.

\begin{figure}
	
	\centering
	\includegraphics[scale=0.46]{schemat_watki}
	\caption{Schemat budowy wątkowej aplikacji. Źródło: [opracowanie własne]} 
	\label{fig:watki}
	
\end{figure}

\section{Graficzny interfejs użytkownika}

Ten podrozdział poświęcony został opisowemu przedstawieniu graficznego interfejsu użytkownika. Podczas jego tworzenia kierowano się tym, aby był on czytelny oraz intuicyjny w obsłudze. 

\section{Drzewo projektu}

\section{Zabezpieczenia aplikacji}

\graphicspath{{./Komora/images/}}

\chapter{Program komputera komory środowiskowej}

Komora środowiskowa jest częścią urządzenia w której zachodzi wzrost badanej struktury biologicznej. Kluczowe dla jej wzrostu jest zachowanie pewnych cech takich jak odpowiednia ilość dostępnej wody w jej podłożu wzrostowym lub odpowiednie oświetlenie. Oprócz tego przydatne może okazać się śledzenie parametrów takich jak temperatura. Komora środowiskowa jest też miejscem docelowym gdzie działanie grawitacji jest eliminowane poprzez klinorotację. Muszą się więc znajdować również w niej czujniki, które jakość symulowanej mikrograwitacji będą monitorować.

\section{Platforma}

Wybór platformy komputera komory środowiskowej został dokonany na etapie tworzenia projektu urządzenia. Wybrany został komputer Raspberry Pi Zero ze względu na niski pobór energii, co ma znaczenie dla natężenia pola magnetycznego w komorze. Posiada on również wbudowany moduł Wi-Fi za pomocą którego zrealizowano połączenie bezprzewodowe. Umożliwia to także zdalną obsługę komputera poprzez standard SSH (ang. \angver{Secure Shell}) w momencie gdy nastąpi awaria oprogramowania. Ze względu na konieczność obsłużenia kamer oraz analogowego czujnika wilgotności podłoża, komputer wyposażono w dodatkowe nakładki, które zwiększają ilość dostępnych portów USB oraz dodają konwerter analogowo-cyfrowy (ang. \angver{Analog to Digital Converter}, ADC) do magistrali I$^2$C. Na komputerze zainstalowano system operacyjny Raspbian Lite, który nie posiada graficznego interfejsu użytkownika w związku z czym ogranicza zużycie jego zasobów.

\section{Rola programu}

Główną rolą programu jest odczytywanie wartości zbieranych przez szereg czujników, a następnie przekazywanie ich do głównej aplikacji w celu wyświetlenia oraz zapisania w pliku. Oprócz tego program kontroluje oświetlacz komory środowiskowej poprzez regulację sygnału PWM podawanego na jego układ sterujący. Program kontroluje dwie sekcje oświetlacza o osobnych sygnałach PWM oraz dodatkowe diody białe służące do doświetlenia układu w momencie wykonywania zdjęcia. W celu wykonania zdjęć do komputera podłączone zostaną dwie kamery USB, które będą wykonywać zdjęcia zgodnie z poleceniami generowanymi w programie. Program oblicza również na bieżąco jakość uzyskanego stanu mikrograwitacji.

\section{Komunikacja bezprzewodowa}

Komunikacja z poziomu komputera komory została w znacznym stopniu zautomatyzowana. Użytkownik musi jedynie zawrzeć spodziewany adres serwera TCP na którym komora będzie go szukać, w pliku konfiguracyjnym. W momencie kiedy taki serwer zostanie uruchomiony z poziomu aplikacj, komputer komory automatycznie go wykryje i zacznie wymieniać informacje. Kiedy serwer nie został uruchomiony komora ponawia próbę połączenia co \SI{5}{s} aż do momentu kiedy ta operacja zakończy się powodzeniem. Jeśli takie połączenie zostaje nagle zerwane, komputer komory powraca do stanu próby połączenia.
\begin{lstlisting}[language=Python,caption={Obsługa połączenia z serwerem TCP.}]
while True:
	while True:
		with socket.socket(socket.AF_INET, socket.SOCK_STREAM) as sc:
		
			sc.settimeout(5)
			try:
				print(f"Attempting connection to {address}")
				sc.connect((address, port))
			except socket.timeout:
				print("Connection timed out.")
				break
			except ConnectionRefusedError:
				index = 0
				means = [0, 0, 0]  # Resetting the gravity averages.
				saturation_measurement_scheduled = False
				print("Connection refused, reconnection attempt in 5s.")
				time.sleep(5)
				break  # Here interrupt the inner forever loop and continue to watch for the connection.
\end{lstlisting}

Kiedy połączenie zostanie pomyślnie nawiązane rozpoczyna się proces wymiany informacji. Na Rys. \ref{fig:ramka danych} przedstawiono schematycznie w jaki sposób zbudowane są ramki danych wymieniane pomiędzy urządzeniami. Pierwsza część ramki składa się z nagłówka o długości \SI{10}{B}, który zawiera informację o tym ile kolejnych bajtów należy odebrać, aby skompletować wiadomość. Jeśli dana paczka została odebrana przez serwer, odsyła on informację o tym czy należy zauktualizować warunki oświetleniowe, co jest determinowane poprzez akcję użytkownika. Jeśli operator nie zmienił ustawień oświetlenia od poprzedniej wymiany danych, to wysyłany jest jedynie komunikat sygnalizujący zachowanie poprzedniego ustawienia. Cykl ten powtarza się tak długo jak istnieje połączenie pomiędzy komorą środowiskową a rdzeniem systemu kontroli.


\begin{figure}[h]
	\centering
	\includegraphics[scale=.5]{ramka}
	\caption{Schemat wiadomości wymienianych pomiędzy urządzeniami. Źródło: [orpacowanie własne]} 
	\label{fig:ramka danych}
\end{figure}


\section{Czujniki i kamery}

W celach uproszczenia starano się dobrać czujniki tak aby wszystkie mogły zostać obsłużone przez taki sam interfejs. Do tego celu wybrano magistralę I$^2$C ze względu na jej dużą popularność oraz dostępność różnego rodzaju czujników. Na tę samą magistralę znalezniono również konwerter analogowo-cyfrowy, którego nie posiada domyślnie wybrana platforma. Oprócz czujników komputer obsługuje również dwie kamery, aby uwiecznić postęp wzrostu struktury. Poniżej umieszczono listę modeli wybranych czujników oraz kamer.
\begin{itemize}
	\item Akcelerometr LIS3DH, magistrala I$^2$C
	\item Termometr MCP9808, magistrala I$^2$C
	\item Konwerter analogowo-cyfrowy ADS1115, magistrala I$^2$C
	\item Moduł kamery OV5648, port USB
	
\end{itemize}

Dla czujników przygotowany został osobny moduł \textbf{sensors.py}, który zawiera definicje klas odpowiedzialnych za tworzenie interfejsu między nimi a komputerem komory środowiskowej. Jako, że wszystkie obiekty czujników muszą posiadać metody odpowiedzialne za wysyłanie do nich danych lub ich odbieranie, zostały one oparte o dziedziczenie z abstrakcyjnej klasy bazowej (ang. \angver{abstract base class}, ABC). Oznacza to iż dzielą one metody o takiej samej nazwie, które docelowo przeznaczone są do tych samych celów, mimo tego iż mogą różnić się zawartością.
\begin{lstlisting}[language=Python, caption={Abstrakcyjna klasa bazowa czujników.}]
class I2CSensor(ABC):

	def __init__(self, address, bus):
		self.address = address
		self.bus = bus
	
	@abstractmethod
	def read(self, *args, **kwargs):
		pass
	
	@abstractmethod
	def enable(self, *args, **kwargs):
		pass
	
	@staticmethod
	@abstractmethod
	def fake_read():
		pass	
\end{lstlisting}
Metody w jakie wyposażono obiekty czujników to wspólny konstruktor \textbf{\_\_init\_\_}, który inicjalizuje obiekt przy okazji przypisując mu otwartą magistralę oraz adres, \textbf{read} zwracającą odczytaną wartość z czujnika, \textbf{enable}, która dokonuje wszystkich operacji koniecznych do poprawnego działania czujnika oraz \textbf{fake\_read}, która emuluje odczyt z czujnika bez konieczności jego podłączenia, co było pomocne w czasie rozwoju programu. Po tak przygotowanej klasie bazowej dziedziczą klasy reprezentujące już konkretne modele czujników, ze względu na różny algorytm odbioru z nich informacji.
\begin{lstlisting}[language=Python, caption={Przykładowa klasa czujnika.}]
class ADS1115ADC(I2CSensor):
	
	__CONVERSION_REG = 0x00
	__CONFIG_REG = 0x01
	__LO_THRESH_REG = 0x02
	__HU_THRESH_REG = 0x03
	
	def read(self, channel):
		# Read channel voltage in reference to ground. ADS1115 also has a differential measuring mode, which
		# has been omitted, but would be trivial to implement.
		
		ls_byte = 0b11100011  # Comparators disabled, 860SPS data rate
		ms_byte = 0b0011 + ((channel + 4) << 4) + 128  # Amplifier gain set to 1, start single conversion, pick channel.
		self.bus.write_i2c_block_data(self.address, ADS1115ADC.__CONFIG_REG, [ms_byte, ls_byte])
		while self.converting_status():
			pass
		data = self.bus.read_i2c_block_data(self.address,ADS1115ADC.__CONVERSION_REG,2)
		reading = (data[0] << 8) + data[1]
		
		return reading
	
	def enable(self):
		pass
	
	@staticmethod
	def fake_read() -> float:
	
		base_read = 50
		
		return float(np.random.normal(base_read, 10, 1))
	
	def converting_status(self):
	
		config = self.bus.read_i2c_block_data(self.address, ADS1115ADC.__CONFIG_REG, 2)
		
		if config[1] >= 128:
			return False
		else:
			return True
\end{lstlisting}

Kamery również obsługiwane są z poziomu programu poprzez wywoływanie poleceń w powłoce systemowej za pomocą biblioteki \textbf{subprocess}. W tym celu przygotowano prosty skrypt \textbf{take\_pic.sh}, który wykonuje zdjęcia za pomocą obu kamer przy użyciu interfejsu Video4Linux. Konieczne jest aby obie kamery posiadały odpowiednie nazwy adapterów portu a, konkretnie \textbf{/dev/videoCam0} oraz \textbf{/dev/videoCam1}. Uzyskiwane jest to za pomocą pliku reguł \textbf{99-name-cameras.rules}, który rozpoznaje unikatowe dla kamer cechy i tworzy odpowiednie dowiązania symboliczne. Plik ten został również umieszczony w repozytorium projektu i należy umieścić go w katalogu \textbf{/etc/udev/rules.d/} w systemie plików komputera komory środowiskowej. Należy zaznaczyć iż numery umieszczone w pliku reguł odpowiadają konkretnym egzamplarzom kamer, a więc jeśli zostaną one zamienione, należy ów kod również zamienić.
\begin{lstlisting}[caption={Plik reguł \textbf{99-name-cameras.rules}.}]
SUBSYSTEM=="video4linux", ENV{ID_REVISION}=="5131",SYMLINK+="videoCam0"
SUBSYSTEM=="video4linux", ENV{ID_REVISION}=="5127",SYMLINK+="videoCam1"
\end{lstlisting}

Na etapie projektu przetestowano kilka dostępnych na rynku czujników wilgotności gleby. Każdy z nich charakteryzował się szybkim postępem korozji elementów wystawionych na działanie wody. Stworzono więc autorski czujnik składający się z dwóch igieł wykonanych ze stali nierdzewnej. Pomiar wilgotności odbywa się poprzez pomiar napięcia układu, w którym układ igieł wbitych w podłoże wzrostowe działa jak rezystor o zmiennym oporze. Aby ograniczyć korozję czujnika do minimum, zasilanie układu jest włączane jedynie na czas pomiaru, trawjący 3s. Na Rys. \ref{fig:czujnik wilg} przedstawiono głowicę wykonanego czujnika wilgotności. Napięcie z układu mierzone jest poprzez dołączony do komputera przetwornik analogowo-cyfrowy. Na Rys. \ref{fig:czujnik wilg sch} widoczny jest prosty schemat elektryczny obwodu czujnika. Zasilanie doprowadzone jest do układu poprzez tranzystor (w tym wypadku jest to fototranzystor zawarty w transoptorze), sterowany z poziomu programu komputera komory środowiskowej. Aby poprawnie interpretować poziomy napięć, konieczny jest pierwszy pomiar napięcia dla głowicy czujnika zanurzonej w wodzie z docelowym stężeniem substancji odżywczych. Dalsze pomiary są następnie przeprowadzane w odniesieniu do tak przygotowanej referencji.

\begin{figure}[h]
	\centering
	\includegraphics[scale=.4]{schemat_wilg}
	\caption{Schemat elektryczny obwodu czujnika. Źródło: [opracowanie własne]} 
	\label{fig:czujnik wilg sch}
\end{figure}

\begin{figure}[h]
	\centering
	\setlength{\fboxsep}{0pt}
	\setlength{\fboxrule}{1pt}
	\fbox{\includegraphics[scale=.15,angle=90]{czujnik_wilg}}
	\caption{Wykonana głowica czujnika wilgotności podłoża. Źródło: [orpacowanie własne]} 
	\label{fig:czujnik wilg}
\end{figure}

\chapter{Podsumowanie}

placeholder

\section{Działanie systemu}

placeholder

\section{Perspektywy rozwoju systemu}

placeholder


%%%%%%%%%%%%%%%%%%%%%%%%%%%%%%%%%%%%%%%%%%
\backmatter
\pagenumbering{Roman}
\stepcounter{stronyPozaNumeracja}
\setcounter{page}{\value{stronyPozaNumeracja}}

\pagestyle{tylkoNumeryStron}

%%%%%%%%%%% bibliografia %%%%%%%%%%%%
\bibliographystyle{plplain}
\bibliography{Bibliografia/bibliografia}
\addcontentsline{toc}{chapter}{Bibliografia}
%%%%%%%%%  DODATKI %%%%%%%%%%%%%%%%%%% 

\begin{appendices} 


\chapter*{Dokumentacja techniczna}
\addcontentsline{toc}{chapter}{Dokumentacja techniczna}

\chapter*{Spis skrótów i symboli}
\addcontentsline{toc}{chapter}{Spis skrótów i symboli}

\begin{itemize}
\item[MCU] mikrokontroler (ang. \ang{microcontroller unit})
\end{itemize}


\chapter*{Zawartość dołączonej płyty}
\addcontentsline{toc}{chapter}{Zawartość dołączonej płyty}

Do pracy dołączona jest płyta CD z~następującą zawartością:
\begin{itemize}
\item praca w~formacie \texttt{pdf},
\item źródła programu,
\item zbiory danych użyte w~eksperymentach.
\end{itemize}

\listoffigures
\addcontentsline{toc}{chapter}{Spis rysunkow}
\listoftables
\addcontentsline{toc}{chapter}{Spis tabel}
	
\end{appendices}


\end{document}


%% Finis coronat opus.
