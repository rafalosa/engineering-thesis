\chapter{Podsumowanie}

Celem pracy inżynierskiej było przygotowanie systemu kontroli klinostatu w celu umożliwienia przeprowadzania z jego pomocą eksperymentów. W czasie budowy systemu starano się zadbać o jego odporność na działania nieświadomego operatora oraz przejrzystość interfejsu, aby ułatwić jego obsługę. W tym, ostatnim, podsumowującym rozdziale przeprowadzona zostanie ocena stworzonego rozwiązania oraz przedstawione zostaną możliwe ścieżki dalszego jego rozwoju.


\section{Ocena działania systemu}

Stworzony system pozwala na łatwą kontrolę parametrów eksperymentu oraz umożliwia udokumentowanie jego przebiegu w formie danych z czujników oraz zdjęć. Sterowanie samym klinostatem zostało szczegółowo przetestowane przez kilka niezależnych osób, a znalezione błędy zostały naprawione. Daje to dużą pewność co do niezawodności tej części systemu. Przetestowano również każdy moduł stworzony na potrzeby oprogramowania komory środowiskowej, zebrano także pierwsze dane grawitacyjne dla kilkunastominutowych cykli. Zadbano o dokładne testy każdej z części programu komputera komory środowiskowej, co daje pewność odnośnie jego poprawnego działania. Przeprowadzono również testy zkoncentrowane na próbie wymuszenia błędu programu komory z poziomu aplikacji, które zakończyły się pomyślnie. Problemem programu jest czętotliwość z jaką zbierane są dane, ponieważ obecnie wynosi ona \SI{5}{Hz}. Spowodowane jest to głównie kolejkami \textbf{Queue} przez które wymieniane są dane między wątkami. Struktury te są dość powolne i to one ograniczają szybkość wymiany informacji. Drugim elementem jest biblioteka matplotlib, która nie pozwala rozłożyć tworzenia oraz aktualizowania wykresów na kilka wątków. Skutkuje to spowolnieniem wątku głównego kiedy wykresów jest zbyt dużo. Ich obecna ilość została uznana za maksymalną przy której interfejs zachowuje pożądany poziom responsywności pomimo wprowadzenia szeregu optymalizacji.

\section{Perspektywy  dalszego rozwoju systemu}

Pomimo iż stworzony system działa poprawnie oraz spełnia wyznaczone dla niego założenia, istnieje wiele aspektów, w których system mógłby zostać usprawiony lub rozszerzony o dodatkową funkcjonalność. W tym podrozdziale przedstawione zostały moje koncepcje na dalszy rozwój tego oprogramowania.\newline

Obecnie system obsługuje wyłącznie jeden klinostat oraz połączenie tylko z jednym komputerem komory środowiskowej. W przypadku większej ilości urządzeń konieczne byłoby uruchomienie kilku niezależnych instancji tego samego programu, co nie jest dobrym rozwiązaniem. Można więc rozszerzyć go o możliwość obsługi kilku urządzeń jednocześnie, wymagałoby to dodatkowego interfejsu, na którym widoczne byłyby podłączone do systemu urządzenia. W takiej sytuacji należałoby również zrezygnować z komunikacji z klinostatem poprzez port USB, ze względu na ich ograniczoną ilość w każdym komputerze. Rozwiązaniem takie problemu są różnego rodzaju moduły komunikacji bezprzewodowej, które łatwo można obsłużyć poprzez magistrale I$^2$C czy SPI z poziomu MCU klinostatu. Pomimo dużego, pozornego stopnia skomplikowania, rozszerzenie to nie byłoby trudne w implementacji ze względu na obiektową konwencję wykonania obecnego programu. Wymagałoby to natomiast zmiany biblioteki tworzenia wykresów, ze względu na jej ograniczenia wspomniane w podrozdziale \ref{watki}.\\

Kolejnym rozszerzeniem byłby moduł sprzęgający eksperyment kontrolny z obecnym eksperymentem przeprowadzanym w klinostacie. Próby kontrolne poddane byłyby dokładnie takim samym warunkom jak główny eksperyment, pomijając klinorotację oraz zewnętrzne warunki magnetyczne. Moduł ten monitorowałby jednocześnie warunki w obu komorach środowiskowych i zapisywał je do jednego pliku. Zmiany ustawień oświetlenia byłyby dzielone przez oba eksperymenty.\\

Do zaprojektowanych przekładni dodano miejsce zamontowania enkoderów obrotowych, co daje możliwość wykrywania kroków gubionych przez silniki oraz dokładne śledzenie pozycji obu stopni swobody klinostatu. Pozwala to na implementację systemu dokładnego pozycjonowania, jeśli pojawi się potrzeba na owy. Nie zdecydowano się go implementować w tej wersji oprogramowania ze względu na to iż nie zostałby on wykorzystany.\\

Teoretycznie system ten również może zostać zaimplementowany do sterowania klinostatmi wysokoobrotowymi, natomiast wymagałoby to kilku, drobnych zmian w kodzie programu. Istnieje możliwość dodania pewnego kroku konfiguracyjnego, zaraz po uruchomieniu programu. Opierałoby się to o poproszenie użytkownika o wprowadzenie parametrów klinostatu takich jak maksymalna prędkość obrotowa, rozdzielczość krokowa silników czy też liczba jednostek napędowych. Program wtedy tworzyłby profil wprowadzonego urządzenia oraz wysłał do sterownika unikatowy identyfikator, który mógłby zostać zapisany w pamięci EEPROM (ang. \angver{Electrically Erasable Programmable Read-Only Memory}) mikrokontrolera. Wczytując wtedy zapisany profil urządzenia, program będzie miał możliwość zidentyfikowania, które urządzenie zostało podłączone. Takie rozwiązanie dałoby możliwość sterowania dowolnym klinostatem, nawet o zupełnie innej konstrukcji bez konieczności ingerowania w kod przez użytkownika.
