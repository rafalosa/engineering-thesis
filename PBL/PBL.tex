\graphicspath{{./PBL/images}}

\chapter{Projekt PBL}

Uczenie poprzez realizację projektów (ang. \angver{Project Based Learning}, PBL) jest metodą uczenia 

\section{Cel projektu} \label{cel_projektu}

Celem wspomnianego projektu było zaprojektowanie układu laboratoryjnego, składającego się z
 klatki Helmholtza oraz umieszczonego wewnątrz klinostatu trójwymiarowego. Docelowo taki układ
  pozwalałby na przeprowadzanie badań nad rozwojem roślin w warunkach symulowanej
   mikrograwitacji bez ziemskiego pola magnetycznego. Klatka Helmholtza jest złożeniem trzech
    par cewek Helmholtza w takiej konfiguracji aby osie magnetyczne każdej z par było
     prostopadłe do pozostałych. Pozwala to wytworzyć wektor indukjcji magnetycznej o dowolnym
      kierunku przestrzennym i bardzo wysokiej jednorodności. Wysoka jednorodność jest kluczowa
       aby generowany wektor indukcji magnetycznej był stały w objętości przeprowadzanego
        eksperymentu. W związku z wymaganiami projektu, wewnętrzna konstrukcja klinostatu
         musiała zostać wykonana tak, aby mieć jak najmniejszy wpływ na panujące wewnątrz
          warunki magnetyczne. Naturalnie rodzi to zagadnienie wyboru materiałów konstrukcyjnych
           klinostatu, które nie powinny mieć właściwości ferromagnetycznych, powinny mieć
            względnie niską przewodność elektryczną w celu eliminacji indukowanych prądów
             wirowych oraz dodatkowym ich atutem będzie niska podatność magnetyczna. Oprócz
              właściwości magnetycznych materiałów istnotne są też możliwości ich obróbki
               termicznej oraz mechanicznej, a również ich forma, co później ma wpływ na
                prostotę montażu urządzenia. Oprócz samych wymagań materiałowych oraz
                 konstrukcyjnych, należało również zapewnić optymalne warunki wzrostowe obiektów
                  badawczych, należało określić odpowiedni rozmiar cewek, aby objętość o wysokim
                   stopniu jednorodności magnetycznej była odpowiednio duża oraz wiele innych.
                    Taki szereg wymagań stworzył bardzo ciekawe i multidyscyplinarne wyzwanie
                     inżynieryjne, które wymagało użycia oraz w niektórych przypadkach
                      stworzenia wielu środowisk symulacyjnych w celu jego rozwiązania. Projekt
                       wykonywany był w zespole 5-cio osobowym na przestrzeni jednego semestru.
                        Powstały na rzecz projektu model komputerowy układu klinostatu wraz z
                         klatką Helmholtza przedstawiony został na Rys.
                          \ref{fig:klatka_helmholtza}.
                          

\begin{figure}
	\centering
	\includegraphics[scale=0.5]{klinostat_klatka}
	\caption{Projekt klatki Helmholtza z klinostatem.} 
	\caption*{Źródło: opracowanie własne}
	\label{fig:klatka_helmholtza}
\end{figure}

\section{Konstrukcja klinostatu}

Ten podrozdział poświęcony został krótkiemu opisowi konstrukcji samego klinostatu
 zaprojektowanego w ramach PBL. Zaprojektowane urządzenie jest klinostatem o dwóch stopniach
  swobody, każdy stopień posiada swój osobny napęd. Oznacza to iż jest to maszyna RPM, natomiast
   jest możliwość uruchomienia go w trybach klinostatu 2-D oraz 3-D jak opisano w podrozdziale
    \ref{klinostat3d}. Zewnętrzna rama klinostatu wykonana została z rur polipropylenowych (PP)
     stabilizowanych włóknem szklanym. Odcinki rur wraz z kształtkami 90$^\circ$ oraz
      czwórnikami zostały połączone metodą polifuzji termicznej (zgrzewanie). Tego typu rury
       wykorzystywane są w instalacjach centralnego ogrzewania, dzięki czemu ich koszt jest
        niski. Konstrukcję ramy przedstawiono na Rys. \ref{fig:rama_klinostatu}. Przeprowadzone
         analizy elementów skończonych (FEM), wskazały iż materiał ten posiada wystarczającą
          wytrzymałość aby wykorzystać go jako element strukturalny ramy klinostatu. Jako wał
           obrotowy wykorzystano rurę aluminiową o średnicy \SI{12}{mm}. Wykorzystanie
            konwencjonalnych łożysk kulowych w konstrukcji klinostatu nie było wskazane ze
             względu na wymagania opisane w podrozdziale \ref{cel_projektu}. Z tego powodu każdy
              z interfejsów obrotowych stanowią tuleje ślizgowe wykonane z materiału
               iglidur$\copyright$, zaprojektowanego przez firmę IGUS. Większość pozostałych
                elementów została wykonana w technologii druku przestrzennego z materiału PETG.
                 Konstrukcję jednego z czterech czwórników ramy przedstawiono na Rys.
                  \ref{fig:czwórnik}. Wewnętrzny stopień swobody składa się z kulistej komory
                   środowiskowej, która posiada swój dedykowany komputer o niskiej mocy,
                    monitorujący panujące wewnątrz warunki, oraz sterujący oświetleniem.
                     Zasilanie do wnętrza komory doprowadzone jest przez szereg złącz
                      ślizgowych, a przewody poprowadzone są wewnątrz ramy klinostatu. Podobne
                       rozwiązanie wykorzystano w obwodzie doprowadzjącym wodę, przewody
                        prowadzone są wewnątrz ramy klinostatu, a przy elementach obrotowych
                         zastosowano kolanka obrotowe. Rozmiar komory oraz typ i moc oświetlenia
                          zostały wyznaczone na podstawie wyników symulacji stworzonych na rzecz
                           projektu. Jej projekt przedstawiony został na Rys. \ref{fig:komora}.
                            Mocowania śrubowe zostały w większości zrealizowane za pomocą śrub
                             poliamidowych (PA). Komora została zaprojektowana tak, aby
                              umożliwić operatorowi dostęp do jej wnętrza bez konieczności jej
                               demontażu z ramy klinostatu.
                               

\begin{figure}
	\centering
	
	\begin{subfigure}[b]{.49\textwidth}
		\centering
		\includegraphics[width=\textwidth]{rama_40_aisass}
		\caption{Rama klinostatu} 
		\label{fig:rama_klinostatu}
	\end{subfigure}
	\hfill%
	\begin{subfigure}[b]{.49\textwidth}
		\centering
		\includegraphics[width=\textwidth]{2_diss}
		\caption{Czwórnik ramy klinostatu.} 
		
		\label{fig:czwórnik}
	\end{subfigure}\vspace{15mm}%
	
	\begin{subfigure}{.8\textwidth}
		\centering
		\includegraphics[scale=0.3]{Komora_tweaked_colors_exploded}
		\caption{Komora środowiskowa.} 
		\label{fig:komora}
	\end{subfigure}

	\caption{Przykładowe części modelu komputerowego projektu.}
	\caption*{Źródło: opracowanie własne}
	
\end{figure}


\section{Modyfikacje projektu}

Jako, że projekt rozpoczął się na początku pandemii COVID-19, niemożliwe było prowadzenie prac
 konstrukcyjnych równolegle z postępem prac modelowych. Spowodowało to zakończenie się projektu
  na fazie ukończonego modelu komputerowego. Konstrukcję całego urządzenia rozpoczęto rok po
   zakończeniu się projektu PBL w zespole dwuosobowym w którego skład wchodziłem. Podczas
    konstrukcji napotkano wiele problemów, które nie zostały przewidziane na etapie prac
     projektowych i wymagały stworzenia nowych elementów urządzenia, bądź zmodyfikowania już
      istniejących komponentów. Oprócz tego wymagane było również stworzenie metod
       konstrukcyjnych tak aby zachować jego kluczowe cechy, oraz aby mogło ono zostać wykonane
        bez użycia kosztownych metod obróbki takich jak wspomagana numerycznie obróbka maszynowa
         (ang. \angver{Computerized Numerical Control}, CNC). Takich poprawek stworzono bardzo
          wiele na drodze konstrukcji urządzenia, natomiast w tym podrozdziale zostaną opisane
           dwie najbardziej kluczowe dla jego poprawnego działania.
           

\subsection{Przeciwwaga ramy klinostatu}

Podczas pierwszych prób uruchomieniowych klinostatu napotkano okresowo pojawiający się moment
 siły, który hamował ruch obrotowy urządzenia przez zbyt duże obciążenie układu napędowego. Za
  jedno ze źródeł tego problemu zidentyfikowano asymetrię obciążenia ramy klinostatu, powodowaną
   układem zmiany kierunku pasa układu napędowego komory środowiskowej, który ulokowany jest na
    rogu ramy. W tym celu na przeciwległym rogu zamontowano przeciwwagę, które generowała moment
     siły o tym samym kierunku, natomiast o przeciwnym zwrocie. Przeciwwaga składa się z
      drukowanego uchwytu oraz kawałka rury PP, która wypełniona jest piaskiem w celu
       zapewnienia odpowiedniej masy. Skonstruowana oraz zamontowana przeciwwaga przedstawiona
        została na Rys. \ref{fig:przeciwwaga}.

\begin{figure}[ht]
	\centering
	\setlength{\fboxsep}{0pt}
	\setlength{\fboxrule}{1pt}
	\fbox{\includegraphics[scale=0.06]{przeciwwaga}}
	\caption{Zamontowana przeciwwaga.} 
	\caption*{Źródło: opracowanie własne}
	\label{fig:przeciwwaga}
\end{figure}

\subsection{Przekładnie walcowe}

Drugą bardzo istotną dla działania klinostatu modfikacją, był projekt i konstrukcja dwóch
 przekładni walcowych, które zwiększają dostępny moment obrotowy układu napędowego. Z uwagi na
  to iż klinostat przeznaczony będzie przedewszystkim do badań nad organizmami roślinnymi, jego
   prędkość obrotowa powinna mieścić się w zakresie od 1 do 2 RPM. W pierwotnej konfiguracji
    klinostat był w stanie osiągnąć znacznie większe prędkości obrotowe, co umożliwiło
     zastosowanie wspomnianych przekładni. Zaprojektowanie przekładnie stosują redukcję obrotów
      silników 4:1. Pozwala to osiągnąć czterokrotnie wyższy moment obrotowy przed głównym kołem
       napędowym klinostatu. Model przekładni widoczny jest na Rys. \ref{fig:projekt
       	 przekładni}. Przekładnia składa się z dwóch kół zębatych o średnicach kolejno $WSTAWIĆ$
         i $WSTAWIĆ$ oraz ilości zębów odpowiednio $WSTAWIĆ$ i $WSTAWIĆ$. Duże koło zębate jest
          łożyskowane w dwóch miejscach przez konwencjonalne łożyska kulowe, ze względu na to iż
           przekładnie znajdują się w wystarczająco dużej odległości od objętości jednorodnego
            pola magnetycznego. Na małym kole zębatym umieszczono dodatkowo wpusty umożliwiajace
             montaż enkoderów, jeśli zajdzie taka potrzeba. Całość zamknięta została w korpusie
              odpowiedzialnym za ochronę kół przez pyłem oraz zapobiegającym wydostaniu się
               smaru na zewnątrz. Skonstruowana przekładna z nałożonym kołem pasowym
                przedstawiona jest na Rys. \ref{fig:gotowa przekładnia}. Zastosowanie przekładni
                 znacznie poprawiło działanie klinostatu, przy jednoczesnym zachowaniu wymaganej
                  prędkości obrotowej urządzenia.       


\begin{figure}
	\centering
	
	\begin{subfigure}[b]{.49\textwidth}
		\centering
		\includegraphics[width=\textwidth]{rama_40_aisass}
		\caption{Projekt przekładni} 
		\label{fig:projekt przekładni}
	\end{subfigure}
	\hfill%
	\begin{subfigure}[b]{.49\textwidth}
		\centering
		\includegraphics[width=\textwidth]{2_diss}
		\caption{Skonstruowana przekładnia.} 
		
		\label{fig:gotowa przekładnia}
	\end{subfigure}

	\caption{Przekładnie klinostatu.}
	\caption*{Źródło: opracowanie własne}
	
\end{figure}



\section{Napęd klinostatu}

\section{Elektronika układu sterowania}